\documentclass[a4paper]{article}
\usepackage{graphicx}
\graphicspath{{./figures/}}
\usepackage[italian]{babel}
\usepackage{float}
\usepackage{braket}
\usepackage{listings}
\usepackage{mdframed}
\usepackage{tikz}
\usepackage{enumitem}
\usetikzlibrary{shapes, arrows, automata, petri, decorations.markings, decorations.pathreplacing, positioning, calc}

\usepackage{hyperref}
\hypersetup{
    colorlinks=false,
}

% Code blocks
\definecolor{codegreen}{rgb}{0,0.6,0}
\definecolor{codegray}{rgb}{0.5,0.5,0.5}
\definecolor{codepurple}{rgb}{0.58,0,0.82}
\definecolor{backcolour}{rgb}{0.95,0.95,0.95}

\lstdefinestyle{mystyle}{
	backgroundcolor=\color{backcolour},
	commentstyle=\color{codegreen},
	keywordstyle=\color{magenta},
	numberstyle=\tiny\color{codegray},
	stringstyle=\color{codepurple},
	basicstyle=\ttfamily\footnotesize,
	breakatwhitespace=false,
	breaklines=true,
	captionpos=b,
	keepspaces=true,
	numbers=left,
	numbersep=5pt,
	showspaces=false,
	showstringspaces=false,
	showtabs=false,
	tabsize=2
}

\lstset{style=mystyle}

\begin{document}

% Title ------------------------------------------------------------------------------
\title{Documentazione progetto Ingegneria del Software\\[1ex]
\large Software di gestione dei pazienti diabetici
}

\author{
\vspace{0.8cm}
Università di Verona\\
Imbriani Paolo - VR500437\\
Irimie Fabio - VR501504
}

\begin{figure}
    \centering
    \includegraphics[width=0.3\textwidth]{UniversityofVerona}
\end{figure}

\maketitle 

\pagebreak
% Title ------------------------------------------------------------------------------

\tableofcontents

\pagebreak

\section{Requisiti e Use Case}

\subsection{Note generali}

Il software che si andrà a sviluppare è un sistema di telemedicina di un servizio clinico per la gestione
di pazienti diabetici. Gli attori principali del sistema sono i \textit{medici} (diabetologi) e i \textit{pazienti}; questi ultimi
hanno credenziali di accesso al sistema fornite dagli amministratori del servizio con cui possono autenticarsi.
Se l'autenticazione va a buon fine allora l'utente verrà indirizzato alla propria \textit{home page} in cui potrà
visualizzare le informazioni relative al loro ruolo. Nel seguente diagramma dei casi d'uso sono rappresentati
i principali attori e le loro interazioni con il sistema. Notare che diamo per scontato che tutti gli autori siano già autenticati
per semplificare il diagramma.

\begin{figure}[H]
	\centering
	\includegraphics[width=0.7\textwidth]{usecase}
	\caption{Diagramma dei casi d'uso}
	\label{fig:usecase}
\end{figure}

\subsection{Casi d'uso del paziente}

Una volta che il paziente si è autenticato ed è entrato all'interno della sua area riservata, egli può
inserire i dati giornalieri relativi alla sua glicemia (prima e dopo ogni pasto); per fare questo
ha bisogno di inserire dati quali: sintomi, rilevazione glicemica, eventuali altre patologie o terapie.

\subsubsection{Inserimento dei dati giornalieri}

\begin{mdframed}
  \textbf{Attore}: Paziente\\
  \textbf{Precondizioni}: Il paziente deve essere autenticato\\
  \textbf{Passi}: 
  \begin{enumerate}[nosep]
    \item Il paziente accede alla sua home page
    \item Il paziente entra dentro l'area di inserimento dei dati giornalieri
    \item  Il paziente inserisce il dato della sua glicemia prima pasto e dopo pasto
    \begin{itemize}
		\item  Il paziente può aprire un ulteriore finestra per inserire i sintomi, le terapie e le patologie, con oppurtuna data 
		\item  Può anche inserire le assunzioni di insulina o qualsiasi farmaco prescritto dal diabetologo, specificandone giorno, ora, farmaco e quantità assunta
	\end{itemize}
	\item Il paziente inserisce la data e ora di rilevazione
	\item Il paziente conferma l'inserimento dei dati
  \end{enumerate}
  \textbf{Postcondizioni}: La rilevazione è inserita 
\end{mdframed}
\noindent
Andiamo a specificare come viene gestito \textbf{l'inserimento dei dati} del paziente: i dati glicemici sono quelli
che il paziente inserisce giornalmente e obbligatori per inviare le rilevazioni. Dopo aver inserito i dati, l'utente può anche:
\begin{itemize}
	\item inserire specifiche sulla patalogia, terapia o sintomi
	\item inserire le assunzioni di insulina o qualsiasi altro farmo prescritto dal diabetologo
\end{itemize}

\begin{figure}[H]
	\centering
	\includegraphics[width=0.7\textwidth]{sdPaziente}
	\caption{Sequence Diagram della rilevazione del paziente}
	\label{fig:sdPaziente}
\end{figure}

\subsection{Casi d'uso del medico}

\subsubsection{Visualizzare i dati del paziente}

\begin{mdframed}
	\textbf{Attore}: Medico\\
	\textbf{Precondizioni}: Il medico deve essere autenticato\\
	\textbf{Passi}: 
	\begin{enumerate}[nosep]
	  \item Il medico accede alla sua area riservata
	  \item Il medico può accedere alla lista dei pazienti
	  \item Il medico può visualizzare i dati del paziente e lo storico delle rilevazioni
	\end{enumerate}
	\textbf{Postcondizioni}: nessuna
  \end{mdframed}
\noindent
Il medico una volta che ha acceduto alla sua area riservata può visualizzare i dati di tutti i pazienti. 

\subsubsection{Modificare i dati del paziente}
\begin{mdframed}
	\textbf{Attore}: Medico\\
	\textbf{Precondizioni}: Il medico deve essere autenticato\\
	\textbf{Passi}: 
	\begin{enumerate}[nosep]
	  \item Il medico accede alla sua area riservata
	  \item Il medico accede alla lista dei pazienti
	  \item Il medico seleziona il paziente che vuole gestire
	  \item Il medico può decidere tra le seguenti opzioni:
	  \begin{itemize}
		\item Aggiungere o modificare la terapia del paziente
		\item Aggiungere o modificare le note di un paziente
	  \end{itemize}
	  \item Il medico conferma le modifiche
	  \item Il sistema aggiorna i dati del paziente
	\end{enumerate}
	\textbf{Postcondizioni}: La modifica è effettuata\\
	\textbf{Sequenza alternativa 1}: il medico può in qualunque momento decidere di annullare le modifiche
	e ritornare alla lista dei pazienti\\
	\textbf{Postcondizioni}: La modifica non è effettuata
  \end{mdframed}
  \noindent

\begin{figure}[H]
	\centering
	\includegraphics[width=0.65\textwidth]{sdMedico}
	\caption{Sequence Diagram del medico}
	\label{fig:sdMedico}
\end{figure}

\subsection{Ricezione delle notifiche}

Il paziente può ricevere notifiche dal sistema; se il pazienta si dimentica di assumere i farmaci
il sistema può inviare una notifica per ricordarglielo. 

\begin{mdframed}
	\textbf{Attore}: Paziente o Medico\\
	\textbf{Precondizioni}: Il paziente o medico deve essere autenticato\\
	\textbf{Passi}: 
	\begin{enumerate}[nosep]
	  \item Il paziente o medico accede alla sua home page
	  \item Il paziente o medico entra dentro la sezione delle notifiche
	  \item  Il paziente o medico può visualizzare:
	  \begin{itemize}
		  \item  Notifiche già lette
		  \item  Notifiche non lette
	  \end{itemize}
	\end{enumerate}
	\textbf{Postcondizioni}: Se viene visualizzata una notifica, questa viene marcata come letta
  \end{mdframed}


\begin{figure}[H]
	\centering
	\includegraphics[width=0.7\textwidth]{sdNotifList.pdf}
	\caption{Sequence Diagram della notifica con lista}
	\label{fig:sdNotifList}
\end{figure}
\noindent
Il sistema invia delle notifiche ai seguenti attori:
\begin{itemize}
	\item \textbf{Al paziente}: per ricordargli di inserire i dati giornalieri
	\item \textbf{Al medico di riferimento del paziente}: per avvisare che il paziente  
	non ha seguito per più di tre giorni consecutivi le prescrizioni.
	\item \textbf{A tutti i medici}: per segnalare i pazienti che registrano
	livelli di glicemia sopra le soglie indicate.
\end{itemize}

\subsection{Casi d'uso dell'amministratore}

L'amministratore del servizio può gestire gli utenti del sistema. Si occupa di creare, modificare e cancellare
gli account dei pazienti e dei medici. Per agevolare l'aggiunta di utenti, ci si può registrare
tramite un form di registrazione (che specificherà il ruolo), ma l'amministratore deve comunque
approvare la registrazione per rendere l'utente attivo nel sistema.

\subsubsection{Gestione dell'utente}

\begin{mdframed}
	\textbf{Attore}: Amministratore\\
	\textbf{Precondizioni}: L'amministratore deve essere autenticato\\
	\textbf{Passi}: 
	\begin{enumerate}[nosep]
	  \item L'amministratore accede alla sua area riservata
	  \item L'amministratore visualizza la lista degli utenti
	  \item L'amministratore può:
	  \begin{itemize}
		  \item  Creare un nuovo utente
		  \item  Modificare un utente
		  \item  Cancellare un utente
	  \end{itemize}
	\end{enumerate}
	\textbf{Postcondizioni}: L'utente è creato, modificato o cancellato
  \end{mdframed}

\subsubsection{Gestione delle richieste di registrazione}

\begin{mdframed}
	\textbf{Attore}: Amministratore\\
	\textbf{Precondizioni}: L'amministratore deve essere autenticato\\
	\textbf{Passi}: 
	\begin{enumerate}[nosep]
	  \item L'amministratore accede alla sua area riservata
	  \item L'amministratore visualizza la lista delle richieste
	  \item Seleziona una richiesta di registrazione
	  \item L'amministratore può:
	  \begin{itemize}
		  \item  Accettare una registrazione
		  \item  Rifiutare una registrazione
	  \end{itemize}
	\end{enumerate}
	\textbf{Postcondizioni}: L'utente è accettato o rifiutato
  \end{mdframed}

\subsection{Visualizzazione della traccia dei medici}

Il sistema tiene traccia di ogni cambiamento o modifica che i medici effettuano sui pazienti, per 
ragioni di sicurezza.

\begin{itemize}
	\item L'amministratore può visualizzare questa traccia per verificare 
	che i medici non stiano effettuando operazioni non autorizzate.
	\item Il medico può visualizzare la traccia per verificare le operazioni effettuate
	precedentemente su un paziente da parte di altri medici
\end{itemize}

\begin{mdframed}
	\textbf{Attore}: Amministratore\\
	\textbf{Precondizioni}: L'amministratore deve essere autenticato\\
	\textbf{Passi}: 
	\begin{enumerate}[nosep]
	  \item L'amministratore accede alla sua area riservata
	  \item L'amministratore accede alla pagina di visualizzazione delle tracce 
	\end{enumerate}
	\textbf{Postcondizioni}: nessuna
\end{mdframed}

\begin{mdframed}
	\textbf{Attore}: Medico\\
	\textbf{Precondizioni}: Il medico deve essere autenticato\\
	\textbf{Passi}: 
	\begin{enumerate}[nosep]
	  \item Il medico accede alla sua area riservata
	  \item Il medico accede alla pagina di visualizzazione dei pazienti
	  \item Il medico seleziona il paziente di cui vuole visualizzare la traccia
	  \item Il medico accede alla schermata di visualizzazione della traccia dell'utente
	\end{enumerate}
	\textbf{Postcondizioni}: nessuna
\end{mdframed}



\end{document}
